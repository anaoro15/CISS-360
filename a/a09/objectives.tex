\textsc{Objectives}
\begin{enumerate}
\li Perform computations involving number representations with fractional parts
in different bases without any restriction on size of the representation 
\li Write C/C++ programs that uses bit programming.
\end{enumerate}

Some general instructions:
\begin{enumerate}

  \li
  Once you have unpacked the download, run \verb!make! to check if your
  \LaTeX\ is missing any files. If your \verb!make! is successful (i.e. no
  error messages), you should see \verb!main.pdf!.
  If there's an error, let me know ASAP.
  
  \li
  The main file is \verb!main.tex!.
  The questions are in the files
  \verb!q01.tex!,
  \verb!q02.tex!, etc.
  You type up your answers in
  \verb!q01s.tex!,
  \verb!q02s.tex!, etc.
  These files are included in \verb!main.tex! by the commands
  such as
  \verb!Convert $123.567$
\begin{enumerate}
\item[(a)] to base $2$
\item[(b)] to base $8$
\item[(c)] to base $16$
\end{enumerate}
with up to at least 8 binary/octal/hexadecimal places.
! (sort of like \verb!#include!).
  So for sure you want to first look at \verb!main.tex!.
  There are also files names \verb!how-to-*.tex! that gives you examples
  on how to typeset certain computations in \LaTeX.
  Look at the examples in these files and make full use of the \LaTeX\ code
  for copy-paste-modify.
  As in all things, you want to work in \lq\lq baby steps" --
  you want to do \verb!make! frequently and check that the pdf generated is OK.


  \li
  I encourage discussion and whiteboarding and checking each others work
  in hangout.
  Or you can discuss on our discord server
  in the ciss360 channel.
  However when you write up the answer,
  it must be your own work.
  Sharing of \LaTeX\ work is plagiarism will result in an immediate $-1000\%$    .

  \li
  Show all your work. All computations should be done by hand, i.e.,
  do not use your TI calculators or any calculators.
  And then you typeset in \LaTeX.

  \li
  As for \LaTeX\ specific questions, again goto hangout or chat in discord.
  You can go to my website \url{http://yliow.github.io}
  and look for \verb!latex.pdf! in the Tutorials section --
  but it's a lot easier to chat in hangout or discord.
  
  \li
  As usual submit using alex.
  
\end{enumerate}
