The following is the famous Collatz conjecture:
Let $f(n)$ (where $n$ is a positive integer) be the function
\[
f(n) =
\begin{cases}
n / 2   & \text{if $n$ is even} \\
3n + 1  & \text{if $n$ is odd}
\end{cases}
\]
(Note that $/$ is the integer division,
for instance $15 / 4 = 3$, i.e. it is not real number division).
Then given
any positive integer $n$ (greater than $0$),
with enough applications of function $f$, you will always arrive at
$1$.
For instance if we start with n = 10:
\begin{align*}
f(10) &= 10 / 2 = 5 \\  
f( 5) &= 3(5) + 1 = 16 \\
f(16) &= 16 / 2 = 8 \\
f( 8) &= 8 / 2 = 4 \\
f( 4) &= 4 / 2 = 2 \\
f( 2) &= 2 / 2 = 1
\end{align*}
This 86-year-old conjecture is still an unsolved problem.
You can write a simple function to
compute $f$. Here's the pseudocode:
\begin{console}
function f(n):
    if n is divisible by 2 then
        return n / 2
    otherwise
        return 3 * n + 1
\end{console}
and to count the number of steps to get to 1, we then do the following:
\begin{console}
function steps(n):
    int num_steps = 0;
    while n is not 1:
        n = f(n)
        num_steps = num_steps + 1
    return num_steps
\end{console} 
Of course you can put everything into a single function:
\begin{console}
function steps(n):
    int num_steps = 0;
    while n is not 1:
        if n is divisible by 2 then
            n = n / 2
        else
            n = 3 * n + 1
        num_steps = num_steps + 1
    return num_steps
\end{console}
The goal of this exercise is to use C/C\texttt{++} bit operations to improve
on the speed of the computation.
(There are other optimizations besides using bit operations but we will not
worry about those
techniques here.)

Note that if you know that $n$ is divisible by $4$
you can compute two steps in one.
With this improvement you have this:
\begin{console}
function steps(n):
    int num_steps = 0;
    while n is not 1:
        q = n / 4 and r = n % 4
        if r is 0, then
            n = n / 4
            num_steps = num_steps + 2
        else if r is 1, then
            ...
        else if r is 2, then
            ...
        else if r is 3, then
            ...
    return num_steps
\end{console}

This pseudocode would then allow you to reach 1 (from n) faster.
The goal is to implement this strategy in a C/C\texttt{++} program.
In your code you cannot use \verb!*! or \verb!/! or \verb!%!,
i.e.
you can use the following arithmetic and bitwise operators:
\verb!+!, \verb!-!,
\verb!&!, \verb!|!, \verb!^!, \verb!>>!, \verb!<<! and their augmented
version, i.e.
\verb!+=!, \verb!-=!, \verb!&=!, \verb!|=!, \verb!^=!, \verb!>>=!, \verb!<<=!.
Make your program as efficient as possible based on the above
methods. Do not include methods not described above.
To test your program you should work out by
hand the numbers of steps to reach 1 for n = 1, 2, 3, ..., 20
and run your program against the values
computed by hand.

\textsc{Test 1}
\begin{console}[commandchars=\\\{\}]
\userinput{10}
6
\end{console}
(The user enters 10 and the program prints 6, i.e., after 6 steps we get 1.)

FYI: This is the end of this problem for us, but the story of this conjecture
goes on. The above is not
the only optimization technique relevant to this program.
The point is to show you that basic math and
some bit programming can help in speeding up certain types of computations.

(Put your C\texttt{++} code in \verb!q04s.tex!.)
