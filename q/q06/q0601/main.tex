\newcommand\COURSE{ciss360}
\newcommand\ASSESSMENT{q0306}
\newcommand\ASSESSMENTTYPE{Quiz}
\newcommand\POINTS{\textwhite{xxx/xxx}}

\input{myquizpreamble}
\input{yliow}
\input{\COURSE}
\textwidth=6in

\input{thispackages}
\input{thismacros}

\renewcommand\TITLE{\ASSESSMENTTYPE \ \ASSESSMENT}



\renewcommand\AUTHOR{aoro1@cougars.ccis.edu} % CHANGE TO YOURS

\begin{document}
\topmattertwo

%------------------------------------------------------------------------------
\nextq
Write a MIPS program gets an array of integers from the user and store
the integers in the data segment. 
The array ends when the user enters $-1$;
$-1$ is not part of the list.
For instance if the array of integers is $4, 2, 3, 5, 7$, the user enters
$4, 2, 3, 5, 7, -1$.
(You may assume that the the list contains at least one integer.)
The program then stores the size of the array
and the values of the array in the data segment (at the beginning).

The program then scans the array of integers and prints the minimum value
of the array.

For instance if the users enter $4, 2, 3, 5, 7, -1$, then
$5, 4, 2, 3, 5, 7$ are stored in the data segment and $2$ is printed.

Your code should be similar to a C/C\texttt{++} program that uses a
pointer to scan 
the array of integers instead of an index variable.

\ANSWER
\begin{answercode}
       		.text
        .globl main

main:
        # get array of integers from the user
        addi  $t0, $zero, -1            # initialize index to 0
	la    $t1, numbers          #load address of numbers
input_loop:
        li    $v0, 5
        syscall
	move  $s0, $v0
        beq   $s0, $t0, exit_loop    # terminate when -1 entered
        sw    $s0, 0($t1)           # store  the input value in the array at current index
        addi  $t2, $t2, 1           # increment the size
        addi  $t1, $t1, 4           # move to the next element
        j     input_loop

exit_loop:
        # find and print the minimum value
        la     $t1, numbers         # reset address
        addi   $t2, $t2, -1	    # up to n - 1
        lw     $t3, 0($t1)          # load the first element of the array
        move   $t4, $t3             # initialize min with the first element
        addi   $t1, $t1, 4          # move to the next element

find_min_loop:
        beq    $t2, $zero, print_min  # exit when the entire array is scanned

        lw     $t3, 0($t1)          # load the current element of the array
        blt    $t3, $t4, update_min # if t3 < t4, update min

        addi   $t1, $t1, 4          # move to the next element
        addi   $t2, $t2, -1         # decrement the counter
        j      find_min_loop

update_min:
        move    $t4, $t3        # move t3 to t4
        addi    $t1, $t1, 4
        addi    $t2, $t2, -1
        j       find_min_loop

print_min:
        li      $v0, 1
        move    $a0, $t4
        syscall

        li      $v0, 10
        syscall
        
        .data
numbers: .word 0
\end{answercode}

%------------------------------------------------------------------------------
\newpage
\input{instructions.tex}
\end{document}