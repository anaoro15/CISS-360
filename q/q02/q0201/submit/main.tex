\newcommand\COURSE{ciss360}
\newcommand\ASSESSMENT{q0306}
\newcommand\ASSESSMENTTYPE{Quiz}
\newcommand\POINTS{\textwhite{xxx/xxx}}

\input{myquizpreamble}
\input{yliow}
\input{\COURSE}
\textwidth=6in

\input{thispackages}
\input{thismacros}

\renewcommand\TITLE{\ASSESSMENTTYPE \ \ASSESSMENT}



\renewcommand\AUTHOR{aoro1@cougars.ccis.edu} % CHANGE TO YOURS

\begin{document}
\topmattertwo

%-------------------------------------------------------------------------------
\nextq
Write a MIPS instruction or pseudo-instruction to
store the integer 42 in register \verb!$s0!.
\\
\ANSWER
\begin{answercode}
li      $s0, 42
\end{answercode}

%-------------------------------------------------------------------------------
\nextq
Write a MIPS instruction or pseudo-instruction to
store the integer 1 in register \verb!$t0!.
\\
\ANSWER
\begin{answercode}
li      $t0, 1
\end{answercode}

%-------------------------------------------------------------------------------
\nextq
Write a MIPS instruction or pseudo-instruction to store the integer sum of
the contents of \verb!$s0! and \verb!$t0! in register \verb!$s1!.
\\
\ANSWER
\begin{answercode}
add     $s1, $s0, $t0
\end{answercode}

%-------------------------------------------------------------------------------
\nextq
Write a MIPS instruction or pseudo-instruction to
copy the contents of \verb!$s0! to \verb!$t1!.
\\
\ANSWER
\begin{answercode}
move    $t1, $s0
\end{answercode}

%-------------------------------------------------------------------------------
The next few questions refer to the following MIPS code fragment:
\begin{console}
li     $s0, 1
li     $t0, 3
add    $s0, $s0, $s0
add    $s1, $s0, $t0
\end{console}

%------------------------------------------------------------------------------
\nextq
What is the value of \verb!$s0! at the end of the above code?
\answerbox{2}

%------------------------------------------------------------------------------
\nextq
What is the value of \verb!$s1! at the end of the above code?
\answerbox{5}

%------------------------------------------------------------------------------
\nextq
What is the value of \verb!$t0! at the end of the above code?
\answerbox{3}

%------------------------------------------------------------------------------
\nextq
Write MIPS instructions or pseudo-instructions to swap the values of register 
\verb!$s0! and 
\verb!$s1!. 
You must use the least number of temporary 
\verb!t!-registers  
Do not use any other registers.
(Hint: You only need one temporary \verb!t!-register.)
\\
\ANSWER
\begin{answercode}
move    $t0, $s0
move    $s0, $s1
move     $s1, $t0
\end{answercode}

%------------------------------------------------------------------------------
\newpage
\input{instructions.tex}
\end{document}